%
% 1- The Idea.tex
%
% This LaTeX source file is for section 1- The Idea for
% the Basic Search lab module for the TAILS project under
% Dr. Stephanie August.
%
\documentclass[11pt]{article}
\usepackage{fullpage}
\usepackage{setspace}
\usepackage{doc}
\usepackage{graphicx}
\usepackage{fancyhdr}
%\usepackage{fancyheadings}
\usepackage{lastpage}

\setlength{\headheight}{15.2pt}
\pagestyle{fancy}
\def\thechapter       {\arabic{chapter}}

% Need to work on header/footer still
%\fancyhf{}
%\rhead{\thechapter : \chaptername}
%\lfoot{Loyola Marymount University}
%\cfoot{\thechapter - \thepage}
%\rfoot{\chaptername\newline Basic Search}

\title{Basic Search:
    Implementations of Basic Uninformed and Heuristic Search Algorithms}
\author{Andrew Won}
\date{February 16, 2012}

\begin{document}
\thispagestyle{plain}

\begin{center}
\textbf{\Large
\vspace{11mm}\\
Basic Search:\\
Implementations of Basic Uninformed and Heuristic\\Search Algorithms}
\vspace{11mm}\\
\includegraphics[width=320pt]{3image_cover.jpg}
\vspace{11mm}\\
\textbf{\Large Artificial Intelligence Laboratory}
\vspace{11mm}\\
\framebox{This document is a TAILS module.}
\vspace{15mm}\\
Andrew Won\\
\today\\
Loyola Marymount University
\end{center}

\newpage
\thispagestyle{fancy}

\chapter{The Idea}
\section{Purpose}
The purpose of this lab is to familiarize the experimenter with the basic 
search algorithms of:

\begin{enumerate}
  \item[a] Depth-First Search;
  \item[b] Breadth-First Search;
  \item[c] Non-Deterministic Search;
  \item[d] Best-First Search;
  \item[e] Hill-Climbing Search; and
  \item[f] A* Search.
\end{enumerate}

This lab will further introduce the principle of dynamic programming for the
latter three search types and will introduce the experimenter to the wide 
applicability of search algorithms in general.  The experimenter will be 
provided with an interactive game and a working implementation of a basic 
search algorithm.  While performing experiments in this lab module, an 
experimenter will learn how to make slight modifications to the basic search 
algorithm to progress from a Depth-First Search to A* Search.

\section{Background}
\subsection{BASIC Search}
Basic search algorithms are widely used in artificial intelligence and many 
people use basic search algorithms every day without realizing it.  The 
fundamental purpose of basic search algorithms is to enable an agent to go 
from a current state to a desired state.  The implementation of a basic search 
algorithm often requires tree traversal finding either one, or the most 
efficient, path from the agent's current state, or root node, to its desired 
state, or frontier node.  Different variants of basic search can be used 
depending upon the type of problem and how much one knows about the search 
space within which the basic search algorithm is searching.  We can broadly 
categorize the basic search algorithms as either uninformed, or blind, search 
algorithms or informed, or heuristic, search algorithms.

Common examples of the basic search algorithms are in route-finding problems.  
Two examples we'll explore are maze-solving and node-finding problems.  An
example of maze-solving can be seen in figure 1a and an example of
node-finding, taken from Wilson 1992, is presented in Figure 2a.

\subsection{Route Finding: Uninformed / Blind Search}
An uninformed, or blind, search algorithm knows nothing of the search space and
discovers everything as it traverses each path.  Let us look into each of our
three uninformed algorithms in more detail.

\subsubsection{Depth-First Search}
\begin{center}
% Depth-First Search Maze graphic by Andrew Won
\includegraphics[width=300pt]{maze.jpg}\\
Figure 1a: Solving a Maze: Depth-First
\end{center}

A depth-first search chooses the first path available and traverses that path
until it either reaches its goal or cannot go any further.  This is kind of like
trying to solve a maze and only taking right turns on the first time through. 
Then if you weren't able to reach your goal you go back to the last time you
made a right turn and try taking the left path instead.  You continue to do this
until you reach your goal.  Figure 1a is an example of when depth-first may not
be the best solution, but in instances when most paths lead to the goal and only
one path needs to be found a depth-first search works just fine.

\begin{center}
% Depth-First Search pre-date comic by XKCD
\includegraphics[width=220pt]{dfs.jpg}\\
$[$http://xkcd.com/761/$]$
\end{center}

\subsubsection{Breadth-First Search}
A breadth-first search takes one step towards every
neighboring path or node simultaneously.  As a result, a breadth-first search
does not get as deep as quickly as an alternative basic search would, but it is
always very thorough.  Breadth-first search can often take the longest amount of
time and runs the risk of quickly becoming unmanageable in size.  It does,
however, guarantee the best result if it is possible to reach the goal. 
Breadth-first is useful if there are many paths that reach the goal and you are
looking for the most efficient path.  Breadth-first search is also useful if
there are many different paths the agent can take and you are confident that a
goal is not too deep into at least one of the paths.

\subsubsection{Non-Deterministic Search}
Non-deterministic search introduces non-determinism, or randomness, to the uninformed search methods that we've explored.  Non-deterministic search is similar to depth-first search as it traverses a single path until it either reaches a goal state or cannot go any further.  The key difference between depth-first search and non-deterministic search is that each time the agent makes a decision about where to go next it chooses its next path randomly.  If we look to our maze illustration it would be equivalent to randomly choosing a path each time a crossroad is reached.

Non-deterministic search is still a form of uninformed, or blind search and suffers from all of the inefficiencies related, but offers something known as adversary foiling.  With a depth-first search, an agent could perform very inefficiently if the set of states were created in a way so that the depth-first search would purposefully traverse all of the incorrect paths first before ever arriving at the correct path.  A non-deterministic search would successfully foil, or defeat, such an adversarial set of states.

\subsection{Route Finding: Informed / Heuristic Search}
An informed, or heuristic, search algorithm assumes that you know something
about the search space.  Consider again our example route finding problem, 
taken from Winston 1992, and presented in Figure 2a.  If, for instance, you had a
way of determining exactly how far from your destination you were you could use
that information to make decisions that brought you closest to your goal.
Consider the distances mapped in Figure 2b.

\begin{center}
% Map of lettered points, looks like a finite-state automata
\includegraphics[width=300pt]{map_fsm.jpg}\\
Figure 2a: Route Finding Problem
\cite{winston1992}
\end{center}

\subsubsection{Best-First Search}
Consider if an agent was at starting state ``S'', needed to travel to state
``G'', and did not know how to proceed.  The agent would have to resort to
uninformed or blind basic searches and would be at a serious disadvantage. 
Informed, or heuristic, basic searches assume that you have some knowledge about
the search space you are working in.  This rudimentary knowledge helps the agent
make informed decisions each time it has to pick a path to travel down.

\subsubsection{Hill-Climbing Search}


\subsubsection{A* Search}


\newpage

% Generate the bibliography
\bibliography{references}
\bibliographystyle{plain}

\end{document}
